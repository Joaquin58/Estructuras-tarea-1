\textbf{13.-} Con la gramática de expresiones aritméticas, genera las derivaciones y los árboles de las siguientes expresiones:}

\texttt3 { 
a) \(2 \cdot x + 3 \cdot y + z\)
b) \((x + y) \cdot (a + b)\)
c) \(x + 2 \cdot b + 1\)
}


\subsection*{Derivaciones}

\subsection*{a) \(2 \times x + 3 \times y + z\)}

\textbf{Derivación:}

Iniciamos con \( \text{Tree} \). Descomponemos \( \text{Tree} \) en \( T1 + T2 \). Descomponemos cada término:

\[
T1 = 2 \times x
\]

\[
T2 = 3 \times y + z
\]

Cada operación se deriva de un subárbol:

\[
2 \times x \text{ se deriva como } T1 = ( \text{void} \times \text{void} )
\]

\[
3 \times y \text{ y } z \text{ se derivan de forma similar.}
\]

\vspace*{\fill}
\begin{center}
\begin{minipage}{0.2\linewidth}
\centering
\begin{verbatim}
       +
      / \
     +   T2
    / \  / \
   *   * z
  / \ / \
 2  x 3  y

\end{verbatim}
\end{minipage}
\end{center}
\vspace*{\fill}

\subsection*{b) \((x + y) \times (a + b)\)}

\textbf{Derivación:}

Iniciamos con \( \text{Tree} \). Descomponemos \( \text{Tree} \) en \( T1 \times T2 \). Descomponemos cada término:

\[
T1 = (x + y)
\]

\[
T2 = (a + b)
\]

Cada operación se deriva de un subárbol:

\[
x + y \text{ se deriva como } T1 = (T1 + T2)
\]

\[
a + b \text{ se deriva de forma similar.}
\]
\vspace*{\fill}
\begin{center}
\begin{minipage}{0.2\linewidth}
\centering
\begin{verbatim}
       *
      / \
     +   +
    / \ / \
   x  y a  b

\end{verbatim}
\end{minipage}
\end{center}
\vspace*{\fill}
\subsection*{c) \(x + 2 \times b + 1\)}

\textbf{Derivación:}

Iniciamos con \( \text{Tree} \). Descomponemos \( \text{Tree} \) en \( T1 + T2 \). Descomponemos cada término:

\[
T1 = x
\]

\[
T2 = 2 \times b + 1
\]

Cada operación se deriva de un subárbol:

\[
2 \times b \text{ se deriva como } T1 = ( \text{void} \times \text{void} )
\]

\[
1 \text{ se deriva como una hoja } T2 = \text{void}.
\]

\vspace*{\fill}
\begin{center}
\begin{minipage}{0.2\linewidth}
\centering
\begin{verbatim}
       +
      / \
     +   1
    / \
   x   *
      / \
     2   b
\end{verbatim}
\end{minipage}
\end{center}
\vspace*{\fill}

