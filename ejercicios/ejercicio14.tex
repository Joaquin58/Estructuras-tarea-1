\textbf{14.-}\ Demostrar que para todo natural $a,b$ con $b > 0$ se cumple que la función:

\begingroup
\centering
\begin{equation}
    div(a,b) = \left\lbrace
    \begin{array}{ll}
     (0,a) & \textup{si } a<b\\
     f(div(a-b,b)) & \textup{si } a\geq b
    \end{array}
    \right.
\end{equation}
donde $f(x,r) = (x+1,r)$ cunple que $div(a,b)=(x,r)$ con $a=x\cdot b+r$

\endgroup

\vspace*{1em}
Inducción sobre los pasos del algoritmo.

\textbf{a) Caso base: }\ En $0$ pasos, tenemos que $a<b$ $\therefore$ Regresa $x=0$ y $r=a$. $\therefore$ queremos ver que $a=b\cdot 0 + a = 0+a=a$ y $a = r < b$

\textbf{b) Hip. Ind.: }\ Sup. que el algoritmo en $n$ pasos regresa $c$ y $r$ tal que $a=bx+r$.

\textbf{c) Paso inductivo: }\ P.D. que se cumple para $n+1$ pasos.
Entonces tenemos que aplicar la función $$f(x_n,r_n)=(x+1,r)=x_{n+1},r_{n+1}$$.


Donde 
\begin{align*}
    b\cdot c_{n+1}+r_{n+1}=&b(n+1)+r_{n}-b\\
    =&b(x_n)+b+r_n-b\\
    =&b(x_n)+r_n\\
\textbf{Por H.I}=&a
\end{align*}

$\therefore \text{el algoritmo termina}\Rightarrow \forall a,b\in \mathbb{N}(b>0)$ se cumple la función

