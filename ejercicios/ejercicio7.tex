\textbf{7.-}\ Demostrar que la suma es conmutativa; esto es, que para toda $m,n \in N$ se cumple que: $m + n = n + m$ (se debe probar que $s(m + n) = s(m) + n$, es decir, que la definición de suma también se puede hacer por la izquierda).

Demostración por inducción sobre $n$ para $s(m + n) = s(m) + n$

\textbf{a) Caso base: }\ $n=0$
\begin{align}
    s(m+0)=&s(m)\\
          =&s(m)+0
\end{align}

\textbf{b) Hip. Ind.: }\ Sup. que $s(m + n) = s(m) + n$

\textbf{c) P.D. que se cumple para: }\ $s(m + s(n)) = s(m) + s(n)$
\begin{align}
    s(m+(n+1))=&s((m+n)+1)\\
    =&s(s(m+n))\\
    \textbf{H.I}=&s(s(m)+n)\\
    =&s(m+1)+n\\
    =&(m+1)+1+n\\
    =&m+1+n+1\\
    =&s(m)+s(n) \blacksquare
\end{align}

Por inducción sobre $n$ para $m + n = n + m$

\textbf{a) Caso base: }\ $n=0$
\begin{align}
    m+0=&m=\\
          =0+m
\end{align}

\textbf{b) Hip. Ind.: }\ Sup. que $m + n = n + m$

\textbf{c) P.D. que se cumple para: }\ $m + s(n) = s(n) + m$
\begin{align}
    m+s(n)=&s(n)+m\\
    m+s(n)=&m+(n+1)\\
          =&(m+n)+1\\
          =&s(m+n)\\
    \textbf{H.I}=&s(n+m)\\
    \textbf{Por el inciso anterior}=&s(n)+m\blacksquare
\end{align}
$$\therefore \forall n,m\in \mathbb{N}, m+n=n+m$$
