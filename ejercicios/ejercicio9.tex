\textbf{9.-}\ Defínase la siguiente sucesión recursivamente:
\begin{itemize}
    \setlength{\itemindent}{5em} 
    \item Base $a_1 = 1 y a_2 = 3$
    \item Recursión $a_n=a_{n-1}+2\cdot a_{n-2}$
\end{itemize}
Demostrar que para toda $n$, $a_n$ es impar.

\textbf{a) Caso base: }\ $a_1=1$
\begin{align*}
    a_1=&1\\
          =&2(0)+1\\
    a_2=&3=&2(1)+1
\end{align*}

\textbf{b) Hip. Ind.: }\ Sup. que se cumple para
\begin{align*}
    k\leq n,a_k=&a_{k-1}+2a_{k-2}\\
    a_k =&2m+1
\end{align*}

\textbf{c) Paso inductivo: }\ P.D. que se cumple para: $k+1,a_{k+1} =2m'+1$
\begin{align*}
    a_{k+1}=&a_{(k+1)-1}+2\cdot a_{(k+1)-2}\\
            =&a_k + 2\cdot a_{k-1}\\
    \textbf{Por H.I}=&2m_1 +1 +2(2m_2 +1)\\
                   =&2m_1 +1 +4m_2 +2\\
                    =&2(m_1 + 2m_2 +1) +1\\
\textbf{Notemos que }&(m_1 + 2m_2 +1)\textbf{ es entero}\\
      =&2m'+1\blacksquare
\end{align*}

$$\therefore \textbf{existe un k entero tal que } a_n=2k+1$$
$$\therefore \forall a_n\in \mathbb{N}, a_n=2k+1 \textbf{ es impar}$$