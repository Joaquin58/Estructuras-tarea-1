\textbf{8.-}\ Demostrar que el producto es asociativo: $k\cdot(m\cdot n) = (k\cdot m)\cdot n$, para toda k,m
naturales y$n \geq 1$ (HINT: se usa la propiedad distributiva).

Por inducción sobre $n$ en la propiedad distributiva.
\textbf{a) Caso base: }\ $n=0$
\begin{align*}
    (k+m)0=0\\
          =&0+0\\
          =&k(0)+m(0)
\end{align*}

\textbf{b) Hip. Ind.: }\ Sup. que $(k+m)n=k\cdot n+m\cdot n$

\textbf{c) P.D. que se cumple para: }\ $(k+m)s(n) = k\cdot s(n) + m\cdot s(n)$
\begin{align*}
    (k+m)s(n)=&s(k+m(n+1))\\
            =&(k+m)n+(k+m)\\
    \textbf{Por H.I}=&k\cdot n + m\cdot n + (k+m)\\
     \textbf{Por asociatividad de la suma}=&(k\cdot n + k)+(m\cdot n+m)\\
      =& k(n+1)+m(n+1)\\
      =&k\cdot s(n)+m\cdot s(n)\blacksquare
\end{align*}

Por inducción sobre $n$ para $k\cdot(m\cdot n)=(k\cdot m)\cdot n$

\textbf{a) Caso base: }\ $n=1$
\begin{align*}
    k(m\cdot 1)=&k\cdot m\\
          =&(k\cdot m)1
\end{align*}

\textbf{b) Hip. Ind.: }\ Sup. que $k\cdot(m\cdot n)=(k\cdot m)\cdot n$

\textbf{c) Paso inductivo: }\ P.D. que se cumple para: $k\cdot(m\cdot s(n))=(k\cdot m)\cdot s(n)$
\begin{align*}
    k\cdot(m\cdot s(n))=& k\cdot(m\cdot (n+1))\\
   \textbf{Por distributición con la suma} =&k\cdot(m\cdot n + m )\\
   \textbf{Por distributición con la suma} =&k\cdot(m\cdot n) + k\cdot(m)\\
   \textbf{Por H.I} =&(k\cdot m)\cdot n + k\cdot(m)\\
                =&(k\cdot m)\cdot (n + 1) \blacksquare
\end{align*}
$$\therefore \forall k,m,n\in \mathbb{N}, k\cdot(m\cdot n)=(k\cdot m)\cdot n$$

