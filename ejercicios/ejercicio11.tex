\textbf{11.-}\ Demostrar las propiedades de la función de reverso de lista


\begin{itemize}
    \item[a)] $\text{rev}(l_1 \sqcup l_2) = \text{rev}(l_2) \sqcup \text{rev}(l_1)$
    \item[b)] $\text{rev}(\text{rev}(l))$
\end{itemize}

Inducción estructurada.

\begin{enumerate}
    \item[1)] Base: $l_1 = []$
    \[
    \text{rev}([] \sqcup l_2) = \text{rev}(l_2) = \text{rev}(l_2) \sqcup []
    \]
    Así que se cumple que $\text{rev}([] \sqcup l_2) = \text{rev}(l_2) \sqcup \text{rev}([])$.

    \item[2)] Hipótesis de inducción

    Supongamos que se cumple $\text{rev}(l_1 \sqcup l_2) = \text{rev}(l_2) \sqcup \text{rev}(l_1)$.

    \item[3)] Demostración

    Por demostrar que $a : l_1$:
    \[
    \text{rev}((a : l_1) \sqcup l_2) = \text{rev}(l_2) \sqcup \text{rev}(a : l_1)
    \]
    \[
    \text{rev}((a : l_1) \sqcup l_2) = \text{rev}(a : (l_1 \sqcup l_2))
    \]
    \[
    \text{rev}(a : (l_1 \sqcup l_2)) = \text{rev}(l_1 \sqcup l_2) \sqcup [a]
    \]
    Por hipótesis de inducción:
    \[
    \text{rev}(l_1 \sqcup l_2) = \text{rev}(l_2) \cup \text{rev}(l_1)
    \]

    \[
    \text{rev}(l_2) \sqcup \text{rev}(l_1) \sqcup [a] = \text{rev}(l_2) \sqcup \text{rev}(a : l_1)
    \]
    
    \[
    \text{rev}((a : l_1) \sqcup l_2) = \text{rev}(l_2) \sqcup \text{rev}(a : l_1)
    \]

    \[
    \text{rev}(\text{rev}(l)) = l
    \]
    \item[Base:] $l = []$
    \[
    \text{rev}(\text{rev}([])) = []
    \]

    \item[Hipótesis de inducción]

    Supongamos que la propiedad se cumple para $l$:
    \[
    \text{rev}(\text{rev}(l)) = l
    \]

    \item[Demostración:]

    \[
    \text{rev}(\text{rev}(a : l)) = \text{rev}(\text{rev}(l) \sqcup [a])
    \]
    \[
    \text{rev}(\text{rev}(l) \sqcup [a]) = \text{rev}([a] \sqcup \text{rev}(l)) = \text{rev}(\text{rev}(l)) \sqcup [a]
    \]
    Por hipótesis de inducción:
    \[
    \text{rev}(\text{rev}(l)) = l
    \]

    \[
    l \sqcup [a] = a : l
    \]

    \[
    \text{rev}(\text{rev}(a : l)) = a : l
    \]
\end{enumerate}

