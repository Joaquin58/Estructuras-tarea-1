\textbf{15.} Demuestra que la siguiente funcion recursiva para lista de enteros: \\
\[
max (a:[]) = a
\]
\begin{align*}
max (a:l) = \begin{cases}
    a & \text{si } a \geq max(l) \\
    max(l) & \text{si } a < max(l) \\
\end{cases}
\end{align*}
\newline
Regresa siempre el Valor máximo de la lista.\\
\newline 
Por inducción sobre los pasos del algoritmo. \\
\newline
\textbf{1) Caso Base:}\\
\newline
En 0 pasos $\rightarrow$ max (a:[]) = a \\
Regresa "a" que es el máximo de la lista que solo contiene a "a" \\
% [a] \rightarrow (a:[])$ \\
\newline
\textbf{2) Hipótesis de Inducción:} \\
\newline
Suponemos que el algoritmo sobre cualquier lista $l$ en $n$ pasos regresa $max(l)$, es decir, el valor máximo de la lista. \\
\newline

\textbf{3) Paso Inductivo:} Por Demostrar que se cumple  para $(a:l)$ que la función $max$ regresa el valor máximo.\\
Entonces para $n+1$  pasos, se aplica para la lista.
\[
max(l') = max(a:l)
\]
Debemos evaluar en dos casos:
\begin{align*}
max (a:l) = \begin{cases}
    a & \text{si } a \geq max(l) \\
    max(l) & \text{si } a < max(l) \\
\end{cases}
\end{align*}
Para el Caso en que $a \geq max(l)$ se toma el valor $a$, Puesto que por \textbf[H.I.] sabemos que de la lista $l$ la función $max(l)$ nos dará el valor máximo, al comparar este con $a$ tendremos el caso en que $a$ es mayor, por lo que podamos tomar $a$ y también en el caso en que $a = max(l)$. Así en este caso tenemos el valor máximo de la lista y la función termina. \\
\newline

Para el Caso en que $max(l) > a$ también podemos afirmar que por \textbf[H.I.] se cumple tenemos el valor máximo de la lista $max(a:l)$ puesto que $a$ no es mayor al máximo del resto de la lista, luego la función termina. \\ 
\newline
Se cumple para ambos casos.
\newline
\textbf{Por lo tanto: } Podemos afirmar que el algoritmo $max$ regresa el valor máximo para toda lista $l$.\\
El algoritmo $max(l)$ es parcialmente completo y correcto.

