\textbf{3.-}\ Para todo $n \in N$ se tiene que $2^{2n} -1 $ es múltiplo de $3$.\\
\newline
Demostración por Inducción sobre \textit{n}\
\newline
\textbf{1) Caso Base:}\ \textit{n=0}\
\[
2^{2(0)}-1
\]
\[
2^{0}-1
\]
\[
1-1
\]

\[
\frac{0}{3} = 0
\]
\[
0
\]
\newline
$0$ es múltiplo de 3, se cumple el Caso Base.\\
\newline
\textbf{2) Hipótesis de Inducción:}\  La propiedad se cumple para \textit{n}\
\begin{center}
$2^{2n}-1$ es Múltiplo de 3
\end{center}
\[
\frac{2^{2n} - 1}{3} \in N
\]
\textbf{3) Paso Inductivo:}\  Por demostrar que se cumple para \textit{n+1}\
\[
2^{2(n+1)} - 1
\]
\[
2^{2n+2} -1
\]
\[
(2^{2n})(2^{2}) - 1
\]
\[
(2^{2n})(4) - 1 
\]
\[
(4)(2^{2n}) - 1
\]
\[
(4)(2^{2n}) - 4 + 3
\]
\[
((4)(2^{2n}) - 4) +3
\]
\[
(4)(2^{2n}-1) + 3
\]
\begin{center}
\textbf{Por H.I.} $2^{2n}-1$ es Múltiplo de 3
\end{center}
$4$ por un Multiplo de 3 es Múltiplo de 3 más 3 sigue siendo un número de 3\\
\newline
\textbf{Por lo tanto} Demostramos que se cumple para toda \textit{n} que: 
\begin{center}
$2^{2n}-1$ es Múltiplo de 3
\end{center}

